%%!TEX TS-program = pdflatexmk 

% ####################################################################
% ++++++++++++++++++++++++++++++++++++++++++++++++++++++++++++++++++++
% ====================================================================
% ____________________________________________________________________

\documentclass[a4paper, 11pt, fleqn]{paper}
 % add titlepage to options above
 % book supports \frontmatter, \mainmatter and \backmatter commands.

% --------------------------------------------------------

\include{preamble}

% --------------------------------------------------------

\title{EpiC Tool v1.0~\\Submodule Documentation\\ \today}
\author{Bruno Gon\c calves}
\date{\today}

\begin{document}
\maketitle

% turn of those nasty overfull and underfull hboxes
\hbadness=10000
\hfuzz=50pt

% set the number of sectioning levels that get number and appear in the contents
\setcounter{secnumdepth}{3}
\setcounter{tocdepth}{3}

% ####################################################################

~\\~\\~\\
\tableofcontents

% ####################################################################
\newpage

\section{Single Population Epidemic Model (SPEM)}

\subsection{SPEMShell.py}

\texttt{SPEMShell.py} is Python script that is in charge of running the simulations.
The wrapper takes care of running the executable(s) to 
produce the output files.

When calling it, a single parameter is passed:
the name of the directory in which the input files will be found (a
relative subpath starting from a pre-defined base path).


The directory (let's call it SIMDIR) should contain three files called:

\begin{itemize}
\item \texttt{req.cfg}
\item \texttt{simul.in}
\item \texttt{simul.mdl}
\end{itemize}

The  \texttt{req.cfg} file should contain two lines:

\begin{description}
\item  [RUNS:  (integer)]  $1$ for single/live run, $> 1$ for multiple runs 
\item  [OUTVAL: (string)]   a list of compartment labels, as defined in the model, separated by ";" (semicolon)
\end{description}

OUTVAL defines the output quantity to be displayed on the visualization
client application and is interpreted to mean the sum of all the transitions in to the compartments specified.

When \texttt{SPEMShell.py} is invoked the SIMDIR will contain
only the 3 conf files (req.cfg, simul.in, simul.mdl) and an empty
directory called \texttt{output}

At the end of the simulation (after the last step/travel files have
been saved inside the "frames" subdir) the wrapper should create an
empty file inside SIMDIR called \texttt{run.end}
used, especially for multiple runs, to confirm that all
the computation is finished.

The data files of each computation step, and all aggregates generated can be
found in the \texttt{<SIMDIR>/output} directory, in order to be
accessible for further analysis/visualization.

\subsection{Simulation Output}
\subsubsection*{Multiple runs}
For each simulation successfully ended there will be a directory called:\\

\texttt{<ROOT\_DIRECTORY>/output/}\\

\noindent
where \texttt{<ROOT\_DIRECTORY>} is the simulation directory that was passed to SPEMShell.py.
Inside the \texttt{output\_data} directory there will be, for each of the  $N$ runs, three files called, respectively, \texttt{SPEM.<r>.out.dat.gz},  \texttt{SPEM.<r>.err.dat.gz}  and \texttt{SPEM.<r>.sec.dat.gz} where \texttt{<r>} is a progressive number identifying the run, ranging from $0$ to $N-1$ when $N$ is the number of runs (the value set in the Simulation Requirements).

\paragraph*{SPEM.\texttt{<r>}.out.dat.gz}

These files contain the complete description of the instantaneous populations in each compartment at every time step.

The first column corresponds to the time step, followed one column for each compartment defined in the model. The sequence of compartments is documented in the first line of the file that starts with a comment char ``\#'' followed by the word ``time''. No blank lines are allowed and every line (including the first one) has the same number of columns, where this number varies from model to model. All entries are listed, even ones composed of all $0$'s. 

\paragraph*{SPEM.\texttt{<r>}.err.dat.gz}

These files contain complete information about all the transitions that occurred during the run. Only non-zero values are listed.

Each line contains four columns, a transition ID, the time step, the number $0$, and the number of individuals that followed that transition in that specific time step.  The transition ID is consistent between runs and it listed in the respective \texttt{GLEaM.\texttt{<r>}.out.dat} (notice the lack of the trailing ``.gz'') in an intuitive notation. For example, for a simple SIR model, this might be:

\begin{center}
\begin{quote}
0 (S) -- (I) (I) 0.7\\
1 (I) -\texttt{>} (R) 0.36\\
\end{quote}
\end{center}

where $0.36$ and $0.7$ are the respective rates. The ID number indicates the order by which it was parsed from the model input file and are used simply as a short hand to refer to a given transition.The rate values listed here are the final numerical values used after all the appropriate substitutions and calculations have been done. 

\paragraph*{SPEM.\texttt{<r>}.sec.dat.gz}

These files contain a subset of the information contained in the previous ones. Again, only non-zero values are listed.  Here we list the total number of secondary cases occurring in each time step. This corresponds to a simple aggregation of the corresponding transitions listed in the respective \texttt{SPEM.<r>.err.dat.gz}. These files have a two column format, time step and number of secondary cases.

\paragraph*{comp.\texttt{<c>}.dat}

For each compartment, \texttt{<c>} defined in the model, a file is generated containing the average population and $95\%$ confidence intervals for each time step.

\paragraph*{outVal.dat}

The \texttt{outVal.dat} file contains information relevant to the outVal compartments specified in the simulation requirements.  Each line,  has $7$ columns, timestep,  the total average number of transitions into an outVal compartment followed by the $95\%$ confidence interval. The final $3$ columns are reserved for the cumulative values.

\paragraph*{\texttt{sum.<a>.<t>.dat}}

Finally, a second set of files can also be found in the  \texttt{output} directory. The respective files names are of the form:

\begin{itemize}
\item \texttt{sum.basins.<t>.dat} 
\item \texttt{step-<date>.1.sum.dat}
\end{itemize}
where \texttt{<t>} the time step and \texttt{<date>} is the corresponding date. 

\texttt{sum} files contain three columns, the number $0$, the number of transitions in to the compartments listed in the requirements file as output variables, and it's cumulative value. By aggregating each of these files, we obtain the \texttt{step} files containing one line of $7$ columns, in the same format as the \texttt{outVal.dat} file, with the difference that the first column is always 0.
Here all values are listed (even zeros) and each file will have $N$ lines, where $N$ is the number of runs. These files are an intermediate step in producing the final output shown in the output visualization, but the user might find them useful for other purposes as well.

\subsubsection*{Single Run}

In the case of a single run, all the files retain their formats, except in what concerns CI as these are no longer defined.

\section{Exact Model Integration (ExactEM)}

In this case, the model, defined as above, is converted into the respective partial differential equations that are then integrated numerically using an adaptive Runge-Kutta procedure. 

\subsection{\texttt{ExactShell.py}}

\texttt{ExactShell.py} is a Python script similar in function to \texttt{SPEMShell.py}. The requirements are the same with a few small differences. Since in this case we are dealing with a deterministic, integration, the RUNS parameter is ignored. Also, at this point, OUTVAL is not supported as it is not clear how to implement it in the current integration procedure. Otherwise, the format of the output files is similar to the one defined for \texttt{SPEMShell.py} in the case of single runs.

\section{Epidemic Models on a Network (NetEM)}

For the near future. Currently working out the best format for network input and outputs.

\end{document}

